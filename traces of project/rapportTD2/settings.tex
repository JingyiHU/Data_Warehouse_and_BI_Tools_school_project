\documentclass[a4paper,11pt]{article}
\usepackage[francais]{babel} % Package babel pour le français
\usepackage[T1]{fontenc}    % Package pour les accentuations
\usepackage[utf8]{inputenc}  % Français
\usepackage{textcomp}
\usepackage{lmodern}        % Pour avoir de bonnes polices en pdf
\usepackage{graphicx}       % Indispensable pour les figures
\usepackage{epstopdf}       % Utile pour les figures, résout une erreur
\usepackage{amsmath}        % Environnement pour les maths, permet du mettre du texte dans les équations
\usepackage{geometry}       % Utilisé pour les marges
\usepackage{pst-tree}		% Pour dessiner des arbres
\usepackage{pst-node}
\usepackage{multicol}		% Pour les colonnes
\usepackage{mathtools, bm}  % Typographie pour les ensembles communs
\usepackage{amssymb, bm}    % Typographie pour les ensembles communs
\usepackage{float}          % Pour bien placer les figures, scripts et tableaux
%\usepackage{listings}       % Utilisé pour les scripts
\usepackage{wrapfig}
\usepackage{xparse}
\usepackage{units}
\usepackage{xcolor}
\definecolor{vertclair}{rgb}{0.10,0.55,0.17}
\definecolor{vertfonce}{rgb}{0,0.44,0}
\definecolor{grisclair}{rgb}{0.78,0.78,0.78}
\definecolor{prune}{rgb}{0.65,0.00,0.00}
\definecolor{bleufonce}{rgb}{0.06,0.06,1.00}
\definecolor{violet}{rgb}{0.21,0.18,0.73}
\definecolor{orange}{rgb}{0.93,0.46,0.00}
% Jolie police
\usepackage{fourier}
% mais on garde les ancien ensemble de math de la police de base :
\DeclareSymbolFont{AMSBbb}{U}{msb}{m}{n}
\AtBeginDocument{\DeclareSymbolFontAlphabet{\mathbb}{AMSBbb}}

%%%%%%%%% TIKZ 
\usepackage{tikz}
\usetikzlibrary{calc,trees,positioning,arrows,chains,shapes.geometric,%
    decorations.pathreplacing,decorations.pathmorphing,shapes,%
    matrix,shapes.symbols}

\tikzset{
>=stealth',
  punktchain/.style={
    rectangle, 
    rounded corners, 
    % fill=black!10,
    draw=black, very thick,
    text width=10em, 
    minimum height=3em, 
    text centered, 
    on chain},
  line/.style={draw, thick, <-},
  element/.style={
    tape,
    top color=white,
    bottom color=blue!50!black!60!,
    minimum width=8em,
    draw=blue!40!black!90, very thick,
    text width=10em, 
    minimum height=3.5em, 
    text centered, 
    on chain},
  every join/.style={->, thick,shorten >=1pt},
  decoration={brace},
  tuborg/.style={decorate},
  tubnode/.style={midway, right=2pt},
}

\usepackage[cache=false]{minted}        % Utilisé pour les scripts
\geometry{hmargin=2cm,vmargin=2cm} % Réglages des marges
\usepackage{fancyhdr}		% Pour l'entête et les pieds de page
\pagestyle{fancy}			% Pour l'entête et les pieds de page
\usepackage{hyperref}		% Pour les liens hypertext, sommaire et références
\usepackage[final]{pdfpages} % Pour inclure des .pdf

\renewcommand{\abstractname}{Avant-propos} %Modification du titre de l'abstract
\newcommand{\norme}[1]{\left\Vert #1\right\Vert} %Commande pour la norme euclidienne
\renewcommand{\listoflistingscaption}{Liste des programmes} %Pour changer le titre de la liste des codes
\renewcommand{\listingscaption}{Programme} %Pour changer la légende des codes
\renewcommand{\O}[1]{$\mathcal{O}\left(#1\right)$} %Pour le O(.) de la complexité
\renewcommand{\P}{\mathbb{P}}
\newcommand{\N}{\mathbb{N}}
\newcommand{\Z}{\mathbb{Z}}
\newcommand{\Fp}{\mathbb{F}_p}
\newcommand{\Fq}{\mathbb{F}_q}

\newcommand{\liendur}[1]{\href{#1}{\texttt{#1}}}

\usepackage{xcolor}
\definecolor{shadecolor}{RGB}{200,200,200}
\newcommand{\enonce}[1]{\par\noindent\colorbox{shadecolor}
{\parbox{\dimexpr\textwidth-2\fboxsep\relax}{#1}}}


\newminted{console}{mathescape,

               breaklines = true,
               numbersep=5pt,
               frame=lines,
               framesep=2mm}

\newminted{c}{mathescape,

               breaklines = true,
               numbersep=5pt,
               frame=lines,
               framesep=2mm}

\renewcommand{\headrulewidth}{0.5pt}
\fancyhead[L]{\textsc{NF26 -- Data Warehouse et Décisionnel}}
\fancyhead[R]{\textsc{Compte-Rendu du TD2}}
% Créations de nouvelles commandess
\newcommand{\citer}[1]{\begin{quotation}« \textit{#1} »\end{quotation}}
\newcommand{\citerin}[1]{« \textit{#1} »}
\newcommand{\citerbeg}[1]{\begin{quotation}« \textit{#1}\end{quotation}}
\newcommand{\citermid}[1]{\begin{quotation}\textit{#1}\end{quotation}}
\newcommand{\citerend}[1]{\begin{quotation}\textit{#1} »\end{quotation}}
\newcommand{\todo}[1]{\begin{center}\textbf{\textsc{À être fait ou compléter:}} #1\end{center}}
%\addto\captionsfrench{\renewcommand{\chaptername}{Partie}} % Pour afficher "Partie" à la place de "Chapitre"
\newcommand{\version}[1]{1.0.0}
\newcommand{\cd}[1]{\mintinline{python}{#1}}

\newcommand{\linehorizon}[0]{\begin{center}\rule{\textwidth}{0.5pt}\end{center}}
\newcommand{\question}[2]{\linehorizon\textsc{Question #1} : \textit{#2}\linehorizon}

\newcommand{\znz}{\mathbb{Z}/n\mathbb{Z}}

\NewDocumentCommand{\f}{o}{\IfNoValueTF{#1}{\mathbb{F}}{\mathbb{F}_#1[X]}}
